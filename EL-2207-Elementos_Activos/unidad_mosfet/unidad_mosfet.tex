\documentclass[../main.tex]{subfiles}
\graphicspath{{\subfix{./}}}
\begin{document}
\everymath{\displaystyle}
%-------------------------------
\vspace{0.2in}
\hrule
\vspace{0.1in}
\section{MOSFET}
\hrule
\begin{xltabular}{\textwidth}{|X|X|}
  \hline
  \endhead
  \hline
  \endfoot
  \endlastfoot
  \underline{\textbf{Condensador MOS}}

  subindices: s = superficie silicio, ox = placa de óxido aislante
  \newline\newline
  $\begin{aligned}
       & K_{\text{SiO}_2} = e_{ox} = 3.9                                      \\
       & \phi_M =\chi +E_c - E_F                                              \\
       & \phi_M =\phi_S                                                       \\
       & \phi_{F_\text{Si-N}} = E_F - E_i = k_BT\ln(N_D/n_i)                  \\
       & \phi_n = \text{debe medirse}                                         \\
       & \phi_p = -60\ln(\frac{N_A}{n_i})\si{\mV}                             \\
       & V_{FB} = - (\phi_{n^+} - \phi_p) =\phi_B = 2\phi_s = v_t\ln(N_A/n_i) \\
    \end{aligned}$
  \newline
  \textbf{límite de inversion}
  $$E_{i_\text{superficie}} - E_{i_\text{sustrato}} = 2(E_F - E_{i_\text{sustrato}})$$
  &
  \underline{\textbf{Parámetros Transistor MOSFET}}
  \begin{center}\includegraphics[scale=0.5]{assets/mosfet}\end{center}
  $$C_{ox}' = \frac{\varepsilon_0K_o}{\chi_o} = \frac{\varepsilon_0\varepsilon_{ox}}{t_{ox}}, [\si{\F\per\square\cm}]$$
  $$C_{ox} = \frac{\varepsilon_0 \varepsilon_{ox}}{t_{ox}}\cdot L\cdot W, \text{[\si{\F}]}$$
  $$V_{GB} = V_{GS}$$

  %-----------
  \\
  %--------------
  \begin{center}
    \begin{tabular}{|c|c|c|}
      \firsthline
      \multicolumn{3}{| c |}{\textbf{Para sustrato N}}
      \\
      \hline
      Estado      & $V_G$       & Concent. de portadores n \\
      \hline
      Acumulación & $> V_{FB}$  & $>N_D$                   \\
      Banda plana & $= V_{FB}$  & $=N_D$                   \\
      Vaciamiento & $< V_{FB}$  & $< N_D$                  \\
      Inversión   & $<< V_{FB}$ & $p > N_D$                \\
      \hline
    \end{tabular}\newline\newline\end{center}
  \textbf{Capacitancias}
  $$C_{ox} = \frac{\epsilon_0K_{ox}A_G}{\chi_{ox}}, ~~~~~ C_{ox_{\text{vac}}} =  \frac{C_{ox} C_s}{C_{ox} + C_s}$$
  \textbf{Relación $V_G$ $\phi_s$ (Si P en vaciamiento)}
  $$\chi_{ox} = t_{ox}$$
  $$V_G = \Delta\phi_s + \Delta\phi_{ox}$$
  $$V_G = \phi_s + \frac{K_s}{K_{ox}}\chi_{ox}\sqrt{\frac{2qN_A}{K_s\epsilon_{ox}}}$$

  &
  $$V_{ov} = \begin{cases} \text{NMOS: } V_{GS} - V_{TH}\\\text{PMOS: }V_{SG} - |V_{TH}| \end{cases}$$
  $$\text{coeff. mod. largo ch: } \lambda, ~~~ [\lambda] = \si{\per\V}$$
  $$\text{Parámetro del proceso } K' = \mu_nC_{ox}'$$
  $$\text{Transconductancia del proceso } K = \mu_nC_{ox}'\frac{W}{L}$$
  $$I_{D_0} = I_D\bigg|_{V_{ov}=0}\approx 0.1 \si{\micro\A}\cdot\frac{W}{L}$$
  $$\text{Pendiente subumbral}\begin{cases}
      S = \bigg[\frac{d}{dV_{GS}}\log(I_{DS})\bigg]^{-1} \\
      S = v_t\ln(10)\cdot m                              \\
      m=1+\frac{C_{dep}}{C_{ox}}\end{cases}$$
  \textbf{Valores en condiciones estándar}
  $$S(T=300)  = 60\si{\mV}/\text{década}$$
  $$S(T=273,15) = 80\si{\mV}/\text{década}$$
  \\
  \hline
  \underline{\textbf{N-channel MOSFET}}
  $$\text{Subumbral} \begin{cases}
      V_{ov} < 0 \\
      I_D = I_{D_0}\exp\bigg(\frac{V_{ov}}{S}\ln(10)\bigg)
    \end{cases}$$
  $$\text{Triodo-lineal}\begin{cases}
      V_{ov} > 0                                             \\
      V_{DS} < V_{ov}                                        \\
      I_D = K\bigg(V_{ov}^2V_{DS} -\frac{1}{2}V_{DS}^2\bigg) \\
      R_{ch} = \frac{V_{DS}}{I_D} = (KV_{ov})^{-1}           \\
    \end{cases}$$
  $$\text{Saturación}\begin{cases}
      V_{ov} > 0                                              \\
      V_{DS} > V_{ov}                                         \\
      I_D = \bigg(\frac{K}{2}V_{ov}^2\bigg)(1+\lambda V_{DS}) \\
    \end{cases}$$
  $$g_m = \frac{\partial I_D}{\partial V_{GS}}\bigg|_Q = KV_{ov} = \frac{2I_D}{V_{ov}} = \sqrt{2KI_D}$$
  &
  \underline{\textbf{P-channel MOSFET}}
  $$\text{Subumbral} \begin{cases}
      V_{ov} < 0 \\
      I_D = I_{D_0}\exp\bigg(\frac{V_{ov}}{S}\ln(10)\bigg)
    \end{cases}$$
  $$\text{Triodo-lineal}\begin{cases}
      V_{ov} > 0                                              \\
      V_{SD} < V_{ov}                                         \\
      I_D = K\bigg(V_{ov}^2V_{SD} - \frac{1}{2}V_{SD}^2\bigg) \\
      R_{ch} = \frac{V_{DS}}{I_D} = (KV_{ov})^{-1}            \\
    \end{cases}$$
  $$\text{Saturación}\begin{cases}
      V_{ov} > 0                                              \\
      V_{SD} > V_{ov}                                         \\
      I_D = \bigg(\frac{K}{2}V_{ov}^2\bigg)(1+\lambda V_{SD}) \\
    \end{cases}$$
  $$g_m = \frac{\partial I_D}{\partial V_{SG}}\bigg|_Q = KV_{ov} = \frac{2I_D}{V_{ov}} = \sqrt{2KI_D}$$
  \\
  \hline

  \begin{center}
    \begin{tabular}{|l|c|c|c|}
      \firsthline
      \multicolumn{4}{|c|}{\textbf{Capacitancias parásitas}}                     \\
      \hline
      Región de operación & $C_{GD}$            & $C_{GB}$ & $C_{GS}$            \\
      \hline
                          &                     &          &                     \\
      Subumbral           & $C_{OV}$            & $C_{OX}$ & $C_{OV}$            \\
                          &                     &          &                     \\
      \hline
                          &                     &          &                     \\
      Triodo              & $\frac{1}{2}C_{OX}$ & $C_{OX}$ & $\frac{1}{2}C_{OX}$ \\
                          &                     &          &                     \\
      \hline
                          &                     &          &                     \\
      Saturación          & $C_{OV}$            & $C_{OX}$ & $\frac{2}{3}C_{OX}$ \\
                          &                     &          &                     \\
      \hline
    \end{tabular}\end{center}
  &
  \\
  \hline
\end{xltabular}
%----------------------------------------
\end{document}